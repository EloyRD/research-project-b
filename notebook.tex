
% Default to the notebook output style

    


% Inherit from the specified cell style.




    
\documentclass[11pt]{article}

    
    
    \usepackage[T1]{fontenc}
    % Nicer default font (+ math font) than Computer Modern for most use cases
    \usepackage{mathpazo}

    % Basic figure setup, for now with no caption control since it's done
    % automatically by Pandoc (which extracts ![](path) syntax from Markdown).
    \usepackage{graphicx}
    % We will generate all images so they have a width \maxwidth. This means
    % that they will get their normal width if they fit onto the page, but
    % are scaled down if they would overflow the margins.
    \makeatletter
    \def\maxwidth{\ifdim\Gin@nat@width>\linewidth\linewidth
    \else\Gin@nat@width\fi}
    \makeatother
    \let\Oldincludegraphics\includegraphics
    % Set max figure width to be 80% of text width, for now hardcoded.
    \renewcommand{\includegraphics}[1]{\Oldincludegraphics[width=.8\maxwidth]{#1}}
    % Ensure that by default, figures have no caption (until we provide a
    % proper Figure object with a Caption API and a way to capture that
    % in the conversion process - todo).
    \usepackage{caption}
    \DeclareCaptionLabelFormat{nolabel}{}
    \captionsetup{labelformat=nolabel}

    \usepackage{adjustbox} % Used to constrain images to a maximum size 
    \usepackage{xcolor} % Allow colors to be defined
    \usepackage{enumerate} % Needed for markdown enumerations to work
    \usepackage{geometry} % Used to adjust the document margins
    \usepackage{amsmath} % Equations
    \usepackage{amssymb} % Equations
    \usepackage{textcomp} % defines textquotesingle
    % Hack from http://tex.stackexchange.com/a/47451/13684:
    \AtBeginDocument{%
        \def\PYZsq{\textquotesingle}% Upright quotes in Pygmentized code
    }
    \usepackage{upquote} % Upright quotes for verbatim code
    \usepackage{eurosym} % defines \euro
    \usepackage[mathletters]{ucs} % Extended unicode (utf-8) support
    \usepackage[utf8x]{inputenc} % Allow utf-8 characters in the tex document
    \usepackage{fancyvrb} % verbatim replacement that allows latex
    \usepackage{grffile} % extends the file name processing of package graphics 
                         % to support a larger range 
    % The hyperref package gives us a pdf with properly built
    % internal navigation ('pdf bookmarks' for the table of contents,
    % internal cross-reference links, web links for URLs, etc.)
    \usepackage{hyperref}
    \usepackage{longtable} % longtable support required by pandoc >1.10
    \usepackage{booktabs}  % table support for pandoc > 1.12.2
    \usepackage[inline]{enumitem} % IRkernel/repr support (it uses the enumerate* environment)
    \usepackage[normalem]{ulem} % ulem is needed to support strikethroughs (\sout)
                                % normalem makes italics be italics, not underlines
    

    
    
    % Colors for the hyperref package
    \definecolor{urlcolor}{rgb}{0,.145,.698}
    \definecolor{linkcolor}{rgb}{.71,0.21,0.01}
    \definecolor{citecolor}{rgb}{.12,.54,.11}

    % ANSI colors
    \definecolor{ansi-black}{HTML}{3E424D}
    \definecolor{ansi-black-intense}{HTML}{282C36}
    \definecolor{ansi-red}{HTML}{E75C58}
    \definecolor{ansi-red-intense}{HTML}{B22B31}
    \definecolor{ansi-green}{HTML}{00A250}
    \definecolor{ansi-green-intense}{HTML}{007427}
    \definecolor{ansi-yellow}{HTML}{DDB62B}
    \definecolor{ansi-yellow-intense}{HTML}{B27D12}
    \definecolor{ansi-blue}{HTML}{208FFB}
    \definecolor{ansi-blue-intense}{HTML}{0065CA}
    \definecolor{ansi-magenta}{HTML}{D160C4}
    \definecolor{ansi-magenta-intense}{HTML}{A03196}
    \definecolor{ansi-cyan}{HTML}{60C6C8}
    \definecolor{ansi-cyan-intense}{HTML}{258F8F}
    \definecolor{ansi-white}{HTML}{C5C1B4}
    \definecolor{ansi-white-intense}{HTML}{A1A6B2}

    % commands and environments needed by pandoc snippets
    % extracted from the output of `pandoc -s`
    \providecommand{\tightlist}{%
      \setlength{\itemsep}{0pt}\setlength{\parskip}{0pt}}
    \DefineVerbatimEnvironment{Highlighting}{Verbatim}{commandchars=\\\{\}}
    % Add ',fontsize=\small' for more characters per line
    \newenvironment{Shaded}{}{}
    \newcommand{\KeywordTok}[1]{\textcolor[rgb]{0.00,0.44,0.13}{\textbf{{#1}}}}
    \newcommand{\DataTypeTok}[1]{\textcolor[rgb]{0.56,0.13,0.00}{{#1}}}
    \newcommand{\DecValTok}[1]{\textcolor[rgb]{0.25,0.63,0.44}{{#1}}}
    \newcommand{\BaseNTok}[1]{\textcolor[rgb]{0.25,0.63,0.44}{{#1}}}
    \newcommand{\FloatTok}[1]{\textcolor[rgb]{0.25,0.63,0.44}{{#1}}}
    \newcommand{\CharTok}[1]{\textcolor[rgb]{0.25,0.44,0.63}{{#1}}}
    \newcommand{\StringTok}[1]{\textcolor[rgb]{0.25,0.44,0.63}{{#1}}}
    \newcommand{\CommentTok}[1]{\textcolor[rgb]{0.38,0.63,0.69}{\textit{{#1}}}}
    \newcommand{\OtherTok}[1]{\textcolor[rgb]{0.00,0.44,0.13}{{#1}}}
    \newcommand{\AlertTok}[1]{\textcolor[rgb]{1.00,0.00,0.00}{\textbf{{#1}}}}
    \newcommand{\FunctionTok}[1]{\textcolor[rgb]{0.02,0.16,0.49}{{#1}}}
    \newcommand{\RegionMarkerTok}[1]{{#1}}
    \newcommand{\ErrorTok}[1]{\textcolor[rgb]{1.00,0.00,0.00}{\textbf{{#1}}}}
    \newcommand{\NormalTok}[1]{{#1}}
    
    % Additional commands for more recent versions of Pandoc
    \newcommand{\ConstantTok}[1]{\textcolor[rgb]{0.53,0.00,0.00}{{#1}}}
    \newcommand{\SpecialCharTok}[1]{\textcolor[rgb]{0.25,0.44,0.63}{{#1}}}
    \newcommand{\VerbatimStringTok}[1]{\textcolor[rgb]{0.25,0.44,0.63}{{#1}}}
    \newcommand{\SpecialStringTok}[1]{\textcolor[rgb]{0.73,0.40,0.53}{{#1}}}
    \newcommand{\ImportTok}[1]{{#1}}
    \newcommand{\DocumentationTok}[1]{\textcolor[rgb]{0.73,0.13,0.13}{\textit{{#1}}}}
    \newcommand{\AnnotationTok}[1]{\textcolor[rgb]{0.38,0.63,0.69}{\textbf{\textit{{#1}}}}}
    \newcommand{\CommentVarTok}[1]{\textcolor[rgb]{0.38,0.63,0.69}{\textbf{\textit{{#1}}}}}
    \newcommand{\VariableTok}[1]{\textcolor[rgb]{0.10,0.09,0.49}{{#1}}}
    \newcommand{\ControlFlowTok}[1]{\textcolor[rgb]{0.00,0.44,0.13}{\textbf{{#1}}}}
    \newcommand{\OperatorTok}[1]{\textcolor[rgb]{0.40,0.40,0.40}{{#1}}}
    \newcommand{\BuiltInTok}[1]{{#1}}
    \newcommand{\ExtensionTok}[1]{{#1}}
    \newcommand{\PreprocessorTok}[1]{\textcolor[rgb]{0.74,0.48,0.00}{{#1}}}
    \newcommand{\AttributeTok}[1]{\textcolor[rgb]{0.49,0.56,0.16}{{#1}}}
    \newcommand{\InformationTok}[1]{\textcolor[rgb]{0.38,0.63,0.69}{\textbf{\textit{{#1}}}}}
    \newcommand{\WarningTok}[1]{\textcolor[rgb]{0.38,0.63,0.69}{\textbf{\textit{{#1}}}}}
    
    
    % Define a nice break command that doesn't care if a line doesn't already
    % exist.
    \def\br{\hspace*{\fill} \\* }
    % Math Jax compatability definitions
    \def\gt{>}
    \def\lt{<}
    % Document parameters
    \title{Section One - Recreation of the Fitness Evaluation Model}
    
    
    

    % Pygments definitions
    
\makeatletter
\def\PY@reset{\let\PY@it=\relax \let\PY@bf=\relax%
    \let\PY@ul=\relax \let\PY@tc=\relax%
    \let\PY@bc=\relax \let\PY@ff=\relax}
\def\PY@tok#1{\csname PY@tok@#1\endcsname}
\def\PY@toks#1+{\ifx\relax#1\empty\else%
    \PY@tok{#1}\expandafter\PY@toks\fi}
\def\PY@do#1{\PY@bc{\PY@tc{\PY@ul{%
    \PY@it{\PY@bf{\PY@ff{#1}}}}}}}
\def\PY#1#2{\PY@reset\PY@toks#1+\relax+\PY@do{#2}}

\expandafter\def\csname PY@tok@w\endcsname{\def\PY@tc##1{\textcolor[rgb]{0.73,0.73,0.73}{##1}}}
\expandafter\def\csname PY@tok@c\endcsname{\let\PY@it=\textit\def\PY@tc##1{\textcolor[rgb]{0.25,0.50,0.50}{##1}}}
\expandafter\def\csname PY@tok@cp\endcsname{\def\PY@tc##1{\textcolor[rgb]{0.74,0.48,0.00}{##1}}}
\expandafter\def\csname PY@tok@k\endcsname{\let\PY@bf=\textbf\def\PY@tc##1{\textcolor[rgb]{0.00,0.50,0.00}{##1}}}
\expandafter\def\csname PY@tok@kp\endcsname{\def\PY@tc##1{\textcolor[rgb]{0.00,0.50,0.00}{##1}}}
\expandafter\def\csname PY@tok@kt\endcsname{\def\PY@tc##1{\textcolor[rgb]{0.69,0.00,0.25}{##1}}}
\expandafter\def\csname PY@tok@o\endcsname{\def\PY@tc##1{\textcolor[rgb]{0.40,0.40,0.40}{##1}}}
\expandafter\def\csname PY@tok@ow\endcsname{\let\PY@bf=\textbf\def\PY@tc##1{\textcolor[rgb]{0.67,0.13,1.00}{##1}}}
\expandafter\def\csname PY@tok@nb\endcsname{\def\PY@tc##1{\textcolor[rgb]{0.00,0.50,0.00}{##1}}}
\expandafter\def\csname PY@tok@nf\endcsname{\def\PY@tc##1{\textcolor[rgb]{0.00,0.00,1.00}{##1}}}
\expandafter\def\csname PY@tok@nc\endcsname{\let\PY@bf=\textbf\def\PY@tc##1{\textcolor[rgb]{0.00,0.00,1.00}{##1}}}
\expandafter\def\csname PY@tok@nn\endcsname{\let\PY@bf=\textbf\def\PY@tc##1{\textcolor[rgb]{0.00,0.00,1.00}{##1}}}
\expandafter\def\csname PY@tok@ne\endcsname{\let\PY@bf=\textbf\def\PY@tc##1{\textcolor[rgb]{0.82,0.25,0.23}{##1}}}
\expandafter\def\csname PY@tok@nv\endcsname{\def\PY@tc##1{\textcolor[rgb]{0.10,0.09,0.49}{##1}}}
\expandafter\def\csname PY@tok@no\endcsname{\def\PY@tc##1{\textcolor[rgb]{0.53,0.00,0.00}{##1}}}
\expandafter\def\csname PY@tok@nl\endcsname{\def\PY@tc##1{\textcolor[rgb]{0.63,0.63,0.00}{##1}}}
\expandafter\def\csname PY@tok@ni\endcsname{\let\PY@bf=\textbf\def\PY@tc##1{\textcolor[rgb]{0.60,0.60,0.60}{##1}}}
\expandafter\def\csname PY@tok@na\endcsname{\def\PY@tc##1{\textcolor[rgb]{0.49,0.56,0.16}{##1}}}
\expandafter\def\csname PY@tok@nt\endcsname{\let\PY@bf=\textbf\def\PY@tc##1{\textcolor[rgb]{0.00,0.50,0.00}{##1}}}
\expandafter\def\csname PY@tok@nd\endcsname{\def\PY@tc##1{\textcolor[rgb]{0.67,0.13,1.00}{##1}}}
\expandafter\def\csname PY@tok@s\endcsname{\def\PY@tc##1{\textcolor[rgb]{0.73,0.13,0.13}{##1}}}
\expandafter\def\csname PY@tok@sd\endcsname{\let\PY@it=\textit\def\PY@tc##1{\textcolor[rgb]{0.73,0.13,0.13}{##1}}}
\expandafter\def\csname PY@tok@si\endcsname{\let\PY@bf=\textbf\def\PY@tc##1{\textcolor[rgb]{0.73,0.40,0.53}{##1}}}
\expandafter\def\csname PY@tok@se\endcsname{\let\PY@bf=\textbf\def\PY@tc##1{\textcolor[rgb]{0.73,0.40,0.13}{##1}}}
\expandafter\def\csname PY@tok@sr\endcsname{\def\PY@tc##1{\textcolor[rgb]{0.73,0.40,0.53}{##1}}}
\expandafter\def\csname PY@tok@ss\endcsname{\def\PY@tc##1{\textcolor[rgb]{0.10,0.09,0.49}{##1}}}
\expandafter\def\csname PY@tok@sx\endcsname{\def\PY@tc##1{\textcolor[rgb]{0.00,0.50,0.00}{##1}}}
\expandafter\def\csname PY@tok@m\endcsname{\def\PY@tc##1{\textcolor[rgb]{0.40,0.40,0.40}{##1}}}
\expandafter\def\csname PY@tok@gh\endcsname{\let\PY@bf=\textbf\def\PY@tc##1{\textcolor[rgb]{0.00,0.00,0.50}{##1}}}
\expandafter\def\csname PY@tok@gu\endcsname{\let\PY@bf=\textbf\def\PY@tc##1{\textcolor[rgb]{0.50,0.00,0.50}{##1}}}
\expandafter\def\csname PY@tok@gd\endcsname{\def\PY@tc##1{\textcolor[rgb]{0.63,0.00,0.00}{##1}}}
\expandafter\def\csname PY@tok@gi\endcsname{\def\PY@tc##1{\textcolor[rgb]{0.00,0.63,0.00}{##1}}}
\expandafter\def\csname PY@tok@gr\endcsname{\def\PY@tc##1{\textcolor[rgb]{1.00,0.00,0.00}{##1}}}
\expandafter\def\csname PY@tok@ge\endcsname{\let\PY@it=\textit}
\expandafter\def\csname PY@tok@gs\endcsname{\let\PY@bf=\textbf}
\expandafter\def\csname PY@tok@gp\endcsname{\let\PY@bf=\textbf\def\PY@tc##1{\textcolor[rgb]{0.00,0.00,0.50}{##1}}}
\expandafter\def\csname PY@tok@go\endcsname{\def\PY@tc##1{\textcolor[rgb]{0.53,0.53,0.53}{##1}}}
\expandafter\def\csname PY@tok@gt\endcsname{\def\PY@tc##1{\textcolor[rgb]{0.00,0.27,0.87}{##1}}}
\expandafter\def\csname PY@tok@err\endcsname{\def\PY@bc##1{\setlength{\fboxsep}{0pt}\fcolorbox[rgb]{1.00,0.00,0.00}{1,1,1}{\strut ##1}}}
\expandafter\def\csname PY@tok@kc\endcsname{\let\PY@bf=\textbf\def\PY@tc##1{\textcolor[rgb]{0.00,0.50,0.00}{##1}}}
\expandafter\def\csname PY@tok@kd\endcsname{\let\PY@bf=\textbf\def\PY@tc##1{\textcolor[rgb]{0.00,0.50,0.00}{##1}}}
\expandafter\def\csname PY@tok@kn\endcsname{\let\PY@bf=\textbf\def\PY@tc##1{\textcolor[rgb]{0.00,0.50,0.00}{##1}}}
\expandafter\def\csname PY@tok@kr\endcsname{\let\PY@bf=\textbf\def\PY@tc##1{\textcolor[rgb]{0.00,0.50,0.00}{##1}}}
\expandafter\def\csname PY@tok@bp\endcsname{\def\PY@tc##1{\textcolor[rgb]{0.00,0.50,0.00}{##1}}}
\expandafter\def\csname PY@tok@fm\endcsname{\def\PY@tc##1{\textcolor[rgb]{0.00,0.00,1.00}{##1}}}
\expandafter\def\csname PY@tok@vc\endcsname{\def\PY@tc##1{\textcolor[rgb]{0.10,0.09,0.49}{##1}}}
\expandafter\def\csname PY@tok@vg\endcsname{\def\PY@tc##1{\textcolor[rgb]{0.10,0.09,0.49}{##1}}}
\expandafter\def\csname PY@tok@vi\endcsname{\def\PY@tc##1{\textcolor[rgb]{0.10,0.09,0.49}{##1}}}
\expandafter\def\csname PY@tok@vm\endcsname{\def\PY@tc##1{\textcolor[rgb]{0.10,0.09,0.49}{##1}}}
\expandafter\def\csname PY@tok@sa\endcsname{\def\PY@tc##1{\textcolor[rgb]{0.73,0.13,0.13}{##1}}}
\expandafter\def\csname PY@tok@sb\endcsname{\def\PY@tc##1{\textcolor[rgb]{0.73,0.13,0.13}{##1}}}
\expandafter\def\csname PY@tok@sc\endcsname{\def\PY@tc##1{\textcolor[rgb]{0.73,0.13,0.13}{##1}}}
\expandafter\def\csname PY@tok@dl\endcsname{\def\PY@tc##1{\textcolor[rgb]{0.73,0.13,0.13}{##1}}}
\expandafter\def\csname PY@tok@s2\endcsname{\def\PY@tc##1{\textcolor[rgb]{0.73,0.13,0.13}{##1}}}
\expandafter\def\csname PY@tok@sh\endcsname{\def\PY@tc##1{\textcolor[rgb]{0.73,0.13,0.13}{##1}}}
\expandafter\def\csname PY@tok@s1\endcsname{\def\PY@tc##1{\textcolor[rgb]{0.73,0.13,0.13}{##1}}}
\expandafter\def\csname PY@tok@mb\endcsname{\def\PY@tc##1{\textcolor[rgb]{0.40,0.40,0.40}{##1}}}
\expandafter\def\csname PY@tok@mf\endcsname{\def\PY@tc##1{\textcolor[rgb]{0.40,0.40,0.40}{##1}}}
\expandafter\def\csname PY@tok@mh\endcsname{\def\PY@tc##1{\textcolor[rgb]{0.40,0.40,0.40}{##1}}}
\expandafter\def\csname PY@tok@mi\endcsname{\def\PY@tc##1{\textcolor[rgb]{0.40,0.40,0.40}{##1}}}
\expandafter\def\csname PY@tok@il\endcsname{\def\PY@tc##1{\textcolor[rgb]{0.40,0.40,0.40}{##1}}}
\expandafter\def\csname PY@tok@mo\endcsname{\def\PY@tc##1{\textcolor[rgb]{0.40,0.40,0.40}{##1}}}
\expandafter\def\csname PY@tok@ch\endcsname{\let\PY@it=\textit\def\PY@tc##1{\textcolor[rgb]{0.25,0.50,0.50}{##1}}}
\expandafter\def\csname PY@tok@cm\endcsname{\let\PY@it=\textit\def\PY@tc##1{\textcolor[rgb]{0.25,0.50,0.50}{##1}}}
\expandafter\def\csname PY@tok@cpf\endcsname{\let\PY@it=\textit\def\PY@tc##1{\textcolor[rgb]{0.25,0.50,0.50}{##1}}}
\expandafter\def\csname PY@tok@c1\endcsname{\let\PY@it=\textit\def\PY@tc##1{\textcolor[rgb]{0.25,0.50,0.50}{##1}}}
\expandafter\def\csname PY@tok@cs\endcsname{\let\PY@it=\textit\def\PY@tc##1{\textcolor[rgb]{0.25,0.50,0.50}{##1}}}

\def\PYZbs{\char`\\}
\def\PYZus{\char`\_}
\def\PYZob{\char`\{}
\def\PYZcb{\char`\}}
\def\PYZca{\char`\^}
\def\PYZam{\char`\&}
\def\PYZlt{\char`\<}
\def\PYZgt{\char`\>}
\def\PYZsh{\char`\#}
\def\PYZpc{\char`\%}
\def\PYZdl{\char`\$}
\def\PYZhy{\char`\-}
\def\PYZsq{\char`\'}
\def\PYZdq{\char`\"}
\def\PYZti{\char`\~}
% for compatibility with earlier versions
\def\PYZat{@}
\def\PYZlb{[}
\def\PYZrb{]}
\makeatother


    % Exact colors from NB
    \definecolor{incolor}{rgb}{0.0, 0.0, 0.5}
    \definecolor{outcolor}{rgb}{0.545, 0.0, 0.0}



    
    % Prevent overflowing lines due to hard-to-break entities
    \sloppy 
    % Setup hyperref package
    \hypersetup{
      breaklinks=true,  % so long urls are correctly broken across lines
      colorlinks=true,
      urlcolor=urlcolor,
      linkcolor=linkcolor,
      citecolor=citecolor,
      }
    % Slightly bigger margins than the latex defaults
    
    \geometry{verbose,tmargin=1in,bmargin=1in,lmargin=1in,rmargin=1in}
    
    

    \begin{document}
    
    
    \maketitle
    
    

    
    \hypertarget{research-project-b}{%
\section{RESEARCH PROJECT B}\label{research-project-b}}

Student: Mr Eloy Ruiz Donayre\\
Supervisor: Prof Dr Achim Kehrein

Reference Paper:\\
\textgreater{} To sleep or not to sleep: the ecology of sleep in
artificial organisms\\
\textgreater{} Alberto Acerbi, Patrick McNamara and Charles L Nunn\\
\textgreater{} https://doi.org/10.1186/1472-6785-8-10

    \hypertarget{experiment-proposed-by-the-paper}{%
\subsection{Experiment proposed by the
paper}\label{experiment-proposed-by-the-paper}}

The paper proposes the creation of artifical individuals with genetic
encoding of its circadian rythm. This genetic enconding corresponds to
24 genes that define the objective for each of the 24 hours of a day,
said objectives could be
\texttt{\textquotesingle{}to\ eat\textquotesingle{}},
\texttt{\textquotesingle{}to\ sleep\textquotesingle{}},
\texttt{\textquotesingle{}attend\ the\ need\ with\ lower\ satisfaction\textquotesingle{}}.
The individual also has an food energy and sleep energy indicator.

Each individual is located in a different grid, each consisting of an
rectangular array of cells with hard boundaries. Each cell of the grid
can be empty or contain one of two kind of resource agents:
\texttt{Sleep\ patches} or \texttt{Food\ patches}.

The agent roams in the grid with memory of its direction (every turn the
direction can be modified by +15, 0, or -15 degrees), the roaming
movement is of one cell by timestep with each timestep representing one
minute. During this roaming movement, both energy and sleep energy are
reduced.

If the objective of the current hour of the day coincides with the
resource available in the cell, the agent won't move and the
corresponding energy indicator of the agent will be increased.

For each experiment run, a population of a 100 individuals was created
with random arrangement of genes and each individual was simulated for
an ammount of timesteps representing 7 days. After this period, all the
agents are ranked by the average of their sleep and food energies and
the 20 fittest were selected as parents of the new population. Each
parent produced 5 offsprings, and during (asexual) reproduction there
was a chance (in the experiment, 5\% chance or 1 mutation in average per
every 20 genes) of occurring a mutation of each gene. Each experiment
run consisted of a 100 generations in the reference paper.

    \hypertarget{characterization-of-the-environment}{%
\subsubsection{Characterization of the
environment}\label{characterization-of-the-environment}}

The grid of each individual was set with a square shape of
\texttt{40\ x\ 40\ cells} and populated with 40 sleep and 40 food
patches.

For the population process for each kind of resource a cluster of
available locations is defined, and from each cluster a subgroup of
location is randomly selected to put the required ammount of patches. To
characterize the distance between clusters and the size of each cluster
two variables are proposed: \texttt{intradistance} and
\texttt{interdistance}.

These are not exactly defined in the paper, so they were defined for
this recreation in the following manner, taking in consideration the
literature of simulation of automatas in grids (like Conway's Game of
Life): \textgreater{} \texttt{interdistance}: this is the Manhattan
distance between the centers of the clusters. \textgreater{}
\texttt{intradistance}: this is the Manhattan size of the cluster, or
the maximum Manhattan distance of any resource to the center of its
cluster.

As the landscapes that evaluate the fitness of each individual try to be
statistically similar in each simulation but have different
distributions of food and sleep patches, this constitutes a situation of
dynamic landscapes. There is no mention of this aspect on the reference
paper.

    \hypertarget{energy-dynamics}{%
\subsubsection{Energy dynamics}\label{energy-dynamics}}

An important factor of the experiment is the energy lost in every
roaming movement, and the energy acquired when a desired resource is
reached. The paper mentions three important aspects about the resources:

\begin{itemize}
\tightlist
\item
  After calibration experiments, a ratio of 3:1 between energy won and
  energy lost was stablished as adecuate.
\item
  The sleeping patches do not deplete.
\item
  The food patches deplete after a certain number of timesteps, and the
  energy provided by them decrease linearly with each timestep of
  exploitation.
\end{itemize}

There is no information about the exact ammounts of energy won and lost,
so the following assumptions were made:

\begin{itemize}
\tightlist
\item
  The energy lost during roaming movement is of one unit of each kind.
\item
  The energy won from sleep patches is of three units.
\item
  The energy won from food patches is dependant on the
  \texttt{depletion\ timesteps} indicated on the experiment. To clarify:
  the energy obtained during the first exploitation timestep is higher
  than the energy obtained in the last one, but the total energy that
  can be obtained during the total existence of a food patch is the same
  that can be obtained from a sleep patch during the same ammount of
  timesteps.
\end{itemize}

    \hypertarget{fitness-evaluation}{%
\subsubsection{Fitness evaluation}\label{fitness-evaluation}}

A fitness value is calculated for each energy value of the individual
through a sigmoid function: (1 + exp(-1/100 * energy) ) \^{} -1. In this
way, the fitness is bound between 0 and 1.

The fitness of the individual corresponds to the average of the sleep
fitness and the food fitness.

There is no discussion in the text about the appropriateness of the
sigmoid function and in some point it is mentiones that in some
experiment individuals with final fitness below 0.9 are discarded, in
what seems a rather arbitrary decision.

    \hypertarget{recreation-of-the-agent-based-model-for-fitness-evaluation}{%
\subsection{Recreation of the Agent-Based Model for Fitness
Evaluation}\label{recreation-of-the-agent-based-model-for-fitness-evaluation}}

The recreation of the model was tried in pure C++ with the knowledge
obtained from the module \texttt{Systems\ and\ organizations} but the
handling of pointers proved too tedious.

Through navigation in Agent-Based Simulation websites, I discovered the
Mesa framework that tries to implement standard features for quick
implementation of simulations in Python (similar to Repast in Java).

For both situations, the approach is of Object Oriented Development
(OOD) with each agent being an instance of a Class.

    \hypertarget{model-setup}{%
\subsection{Model setup}\label{model-setup}}

    \begin{Verbatim}[commandchars=\\\{\}]
{\color{incolor}In [{\color{incolor}1}]:} \PY{o}{\PYZpc{}}\PY{k}{matplotlib} inline
        \PY{k+kn}{from} \PY{n+nn}{jupyterthemes} \PY{k}{import} \PY{n}{jtplot}
        \PY{n}{jtplot}\PY{o}{.}\PY{n}{style}\PY{p}{(}\PY{p}{)}
        \PY{c+c1}{\PYZsh{}import seaborn as sns}
        \PY{c+c1}{\PYZsh{}sns.set()}
\end{Verbatim}


    \begin{Verbatim}[commandchars=\\\{\}]
{\color{incolor}In [{\color{incolor}2}]:} \PY{k+kn}{from} \PY{n+nn}{agents} \PY{k}{import} \PY{n}{Animal}\PY{p}{,} \PY{n}{FoodPatch}\PY{p}{,} \PY{n}{SleepPatch}
        \PY{k+kn}{from} \PY{n+nn}{model} \PY{k}{import} \PY{n}{SleepAnimals}
        
        \PY{k+kn}{import} \PY{n+nn}{numpy} \PY{k}{as} \PY{n+nn}{np}
        \PY{k+kn}{import} \PY{n+nn}{matplotlib}
        \PY{k+kn}{import} \PY{n+nn}{matplotlib}\PY{n+nn}{.}\PY{n+nn}{pyplot} \PY{k}{as} \PY{n+nn}{plt}
\end{Verbatim}


    \begin{Verbatim}[commandchars=\\\{\}]
{\color{incolor}In [{\color{incolor}3}]:} \PY{n}{a} \PY{o}{=} \PY{n}{np}\PY{o}{.}\PY{n}{random}\PY{o}{.}\PY{n}{choice}\PY{p}{(}\PY{p}{[}\PY{l+s+s1}{\PYZsq{}}\PY{l+s+s1}{eat}\PY{l+s+s1}{\PYZsq{}} \PY{p}{,} \PY{l+s+s1}{\PYZsq{}}\PY{l+s+s1}{sleep}\PY{l+s+s1}{\PYZsq{}} \PY{p}{,} \PY{l+s+s1}{\PYZsq{}}\PY{l+s+s1}{flex}\PY{l+s+s1}{\PYZsq{}}\PY{p}{]} \PY{p}{,} \PY{l+m+mi}{24}\PY{p}{)}
        \PY{n}{a}
\end{Verbatim}


\begin{Verbatim}[commandchars=\\\{\}]
{\color{outcolor}Out[{\color{outcolor}3}]:} array(['eat', 'sleep', 'sleep', 'sleep', 'sleep', 'eat', 'flex', 'flex',
               'sleep', 'flex', 'sleep', 'eat', 'sleep', 'sleep', 'eat', 'sleep',
               'sleep', 'flex', 'eat', 'sleep', 'sleep', 'eat', 'sleep', 'flex'],
              dtype='<U5')
\end{Verbatim}
            
    Creation of three grids with three values of
\texttt{interdistance\ factors} (0 , 0.35 and 0.7). The
\texttt{interdistance} is calculated as the
\texttt{interdistance\ factor} times
\texttt{maximum\ Manhattan\ distance\ in\ the\ grid} (this is equal to
the width plus the height of the grid)

    \begin{Verbatim}[commandchars=\\\{\}]
{\color{incolor}In [{\color{incolor}4}]:} \PY{n}{ground\PYZus{}1} \PY{o}{=} \PY{n}{SleepAnimals}\PY{p}{(}\PY{l+m+mi}{1}\PY{p}{,} \PY{n}{genome}\PY{o}{=}\PY{n}{a}\PY{p}{,}  \PY{n}{width}\PY{o}{=}\PY{l+m+mi}{40}\PY{p}{,} \PY{n}{height}\PY{o}{=}\PY{l+m+mi}{40}\PY{p}{,} \PY{n}{interdistance\PYZus{}factor}\PY{o}{=}\PY{l+m+mi}{0}\PY{p}{)}
        \PY{n}{ground\PYZus{}2} \PY{o}{=} \PY{n}{SleepAnimals}\PY{p}{(}\PY{l+m+mi}{1}\PY{p}{,} \PY{n}{genome}\PY{o}{=}\PY{n}{a}\PY{p}{,}  \PY{n}{width}\PY{o}{=}\PY{l+m+mi}{40}\PY{p}{,} \PY{n}{height}\PY{o}{=}\PY{l+m+mi}{40}\PY{p}{,} \PY{n}{interdistance\PYZus{}factor}\PY{o}{=}\PY{l+m+mf}{0.35}\PY{p}{)}
        \PY{n}{ground\PYZus{}3} \PY{o}{=} \PY{n}{SleepAnimals}\PY{p}{(}\PY{l+m+mi}{1}\PY{p}{,} \PY{n}{genome}\PY{o}{=}\PY{n}{a}\PY{p}{,}  \PY{n}{width}\PY{o}{=}\PY{l+m+mi}{40}\PY{p}{,} \PY{n}{height}\PY{o}{=}\PY{l+m+mi}{40}\PY{p}{,} \PY{n}{interdistance\PYZus{}factor}\PY{o}{=}\PY{l+m+mf}{0.7}\PY{p}{)}
\end{Verbatim}


    \begin{Verbatim}[commandchars=\\\{\}]
{\color{incolor}In [{\color{incolor}5}]:} \PY{c+c1}{\PYZsh{} Interdistance or Manhattan distance between centers of Patches regions, }
        \PY{c+c1}{\PYZsh{} and intradistance or maximum Manhattan distance to center in each region}
        \PY{p}{(} \PY{n}{ground\PYZus{}1}\PY{o}{.}\PY{n}{interdistance}\PY{p}{,} \PY{n}{ground\PYZus{}1}\PY{o}{.}\PY{n}{intradistance} \PY{p}{)}
        \PY{p}{(} \PY{n}{ground\PYZus{}2}\PY{o}{.}\PY{n}{interdistance}\PY{p}{,} \PY{n}{ground\PYZus{}2}\PY{o}{.}\PY{n}{intradistance} \PY{p}{)}
        \PY{p}{(} \PY{n}{ground\PYZus{}3}\PY{o}{.}\PY{n}{interdistance}\PY{p}{,} \PY{n}{ground\PYZus{}3}\PY{o}{.}\PY{n}{intradistance} \PY{p}{)}
\end{Verbatim}


\begin{Verbatim}[commandchars=\\\{\}]
{\color{outcolor}Out[{\color{outcolor}5}]:} (0, 16)
\end{Verbatim}
            
\begin{Verbatim}[commandchars=\\\{\}]
{\color{outcolor}Out[{\color{outcolor}5}]:} (28, 16)
\end{Verbatim}
            
\begin{Verbatim}[commandchars=\\\{\}]
{\color{outcolor}Out[{\color{outcolor}5}]:} (56, 16)
\end{Verbatim}
            
    \hypertarget{available-locations-or-clusters-for-food-and-sleep-patches}{%
\subsubsection{Available locations (or clusters) for Food and Sleep
patches}\label{available-locations-or-clusters-for-food-and-sleep-patches}}

    For \texttt{ground\ 1} with interdistance = 0.

Food patches in green and sleep patches in lila.

    \begin{Verbatim}[commandchars=\\\{\}]
{\color{incolor}In [{\color{incolor}6}]:} \PY{n}{available\PYZus{}spots} \PY{o}{=} \PY{n}{ground\PYZus{}1}\PY{o}{.}\PY{n}{arrayRGB\PYZus{}clusters}\PY{p}{(}\PY{p}{)}
        \PY{n}{\PYZus{}} \PY{o}{=} \PY{n}{plt}\PY{o}{.}\PY{n}{figure}\PY{p}{(} \PY{n}{figsize} \PY{o}{=} \PY{p}{(}\PY{l+m+mi}{7}\PY{p}{,} \PY{l+m+mi}{7}\PY{p}{)} \PY{p}{)}
        \PY{n}{\PYZus{}} \PY{o}{=} \PY{n}{plt}\PY{o}{.}\PY{n}{imshow}\PY{p}{(}\PY{n}{np}\PY{o}{.}\PY{n}{swapaxes}\PY{p}{(}\PY{n}{available\PYZus{}spots}\PY{p}{,}\PY{l+m+mi}{0}\PY{p}{,}\PY{l+m+mi}{1}\PY{p}{)}\PY{p}{,} \PY{n}{origin} \PY{o}{=} \PY{l+s+s1}{\PYZsq{}}\PY{l+s+s1}{lower}\PY{l+s+s1}{\PYZsq{}}\PY{p}{)}
        \PY{n}{\PYZus{}} \PY{o}{=} \PY{n}{plt}\PY{o}{.}\PY{n}{plot}\PY{p}{(}\PY{n}{ground\PYZus{}1}\PY{o}{.}\PY{n}{fp\PYZus{}center\PYZus{}x}\PY{p}{,} \PY{n}{ground\PYZus{}1}\PY{o}{.}\PY{n}{fp\PYZus{}center\PYZus{}y}\PY{p}{,} \PY{l+s+s1}{\PYZsq{}}\PY{l+s+s1}{+}\PY{l+s+s1}{\PYZsq{}}\PY{p}{,} \PY{n}{mew}\PY{o}{=}\PY{l+m+mi}{5}\PY{p}{,} \PY{n}{ms}\PY{o}{=}\PY{l+m+mi}{20}\PY{p}{)}
        \PY{n}{\PYZus{}} \PY{o}{=} \PY{n}{plt}\PY{o}{.}\PY{n}{plot}\PY{p}{(}\PY{n}{ground\PYZus{}1}\PY{o}{.}\PY{n}{sp\PYZus{}center\PYZus{}x}\PY{p}{,} \PY{n}{ground\PYZus{}1}\PY{o}{.}\PY{n}{sp\PYZus{}center\PYZus{}y}\PY{p}{,} \PY{l+s+s1}{\PYZsq{}}\PY{l+s+s1}{+}\PY{l+s+s1}{\PYZsq{}}\PY{p}{,} \PY{n}{mew}\PY{o}{=}\PY{l+m+mi}{5}\PY{p}{,} \PY{n}{ms}\PY{o}{=}\PY{l+m+mi}{20}\PY{p}{)}
\end{Verbatim}


    \begin{center}
    \adjustimage{max size={0.9\linewidth}{0.9\paperheight}}{output_15_0.png}
    \end{center}
    { \hspace*{\fill} \\}
    
    \begin{Verbatim}[commandchars=\\\{\}]
{\color{incolor}In [{\color{incolor}7}]:} \PY{c+c1}{\PYZsh{} Coordinate of center of Food Patches Region}
        \PY{p}{(}\PY{n}{ground\PYZus{}1}\PY{o}{.}\PY{n}{fp\PYZus{}center\PYZus{}x}\PY{p}{,} \PY{n}{ground\PYZus{}1}\PY{o}{.}\PY{n}{fp\PYZus{}center\PYZus{}y}\PY{p}{)}
        \PY{c+c1}{\PYZsh{} Coordinate of center of Sleep Patches Region}
        \PY{p}{(}\PY{n}{ground\PYZus{}1}\PY{o}{.}\PY{n}{sp\PYZus{}center\PYZus{}x}\PY{p}{,} \PY{n}{ground\PYZus{}1}\PY{o}{.}\PY{n}{sp\PYZus{}center\PYZus{}y}\PY{p}{)}
\end{Verbatim}


\begin{Verbatim}[commandchars=\\\{\}]
{\color{outcolor}Out[{\color{outcolor}7}]:} (28, 10)
\end{Verbatim}
            
\begin{Verbatim}[commandchars=\\\{\}]
{\color{outcolor}Out[{\color{outcolor}7}]:} (28, 10)
\end{Verbatim}
            
    For \texttt{ground\ 2} with interdistance = 0.35

    \begin{Verbatim}[commandchars=\\\{\}]
{\color{incolor}In [{\color{incolor}8}]:} \PY{n}{available\PYZus{}spots} \PY{o}{=} \PY{n}{ground\PYZus{}2}\PY{o}{.}\PY{n}{arrayRGB\PYZus{}clusters}\PY{p}{(}\PY{p}{)}
        \PY{n}{\PYZus{}} \PY{o}{=} \PY{n}{plt}\PY{o}{.}\PY{n}{figure}\PY{p}{(} \PY{n}{figsize} \PY{o}{=} \PY{p}{(}\PY{l+m+mi}{7}\PY{p}{,} \PY{l+m+mi}{7}\PY{p}{)} \PY{p}{)}
        \PY{n}{\PYZus{}} \PY{o}{=} \PY{n}{plt}\PY{o}{.}\PY{n}{imshow}\PY{p}{(}\PY{n}{np}\PY{o}{.}\PY{n}{swapaxes}\PY{p}{(}\PY{n}{available\PYZus{}spots}\PY{p}{,}\PY{l+m+mi}{0}\PY{p}{,}\PY{l+m+mi}{1}\PY{p}{)}\PY{p}{,} \PY{n}{origin} \PY{o}{=} \PY{l+s+s1}{\PYZsq{}}\PY{l+s+s1}{lower}\PY{l+s+s1}{\PYZsq{}}\PY{p}{)}
        \PY{n}{\PYZus{}} \PY{o}{=} \PY{n}{plt}\PY{o}{.}\PY{n}{plot}\PY{p}{(}\PY{n}{ground\PYZus{}2}\PY{o}{.}\PY{n}{fp\PYZus{}center\PYZus{}x}\PY{p}{,} \PY{n}{ground\PYZus{}2}\PY{o}{.}\PY{n}{fp\PYZus{}center\PYZus{}y}\PY{p}{,} \PY{l+s+s1}{\PYZsq{}}\PY{l+s+s1}{+}\PY{l+s+s1}{\PYZsq{}}\PY{p}{,} \PY{n}{mew}\PY{o}{=}\PY{l+m+mi}{5}\PY{p}{,} \PY{n}{ms}\PY{o}{=}\PY{l+m+mi}{20}\PY{p}{)}
        \PY{n}{\PYZus{}} \PY{o}{=} \PY{n}{plt}\PY{o}{.}\PY{n}{plot}\PY{p}{(}\PY{n}{ground\PYZus{}2}\PY{o}{.}\PY{n}{sp\PYZus{}center\PYZus{}x}\PY{p}{,} \PY{n}{ground\PYZus{}2}\PY{o}{.}\PY{n}{sp\PYZus{}center\PYZus{}y}\PY{p}{,} \PY{l+s+s1}{\PYZsq{}}\PY{l+s+s1}{+}\PY{l+s+s1}{\PYZsq{}}\PY{p}{,} \PY{n}{mew}\PY{o}{=}\PY{l+m+mi}{5}\PY{p}{,} \PY{n}{ms}\PY{o}{=}\PY{l+m+mi}{20}\PY{p}{)}
\end{Verbatim}


    \begin{center}
    \adjustimage{max size={0.9\linewidth}{0.9\paperheight}}{output_18_0.png}
    \end{center}
    { \hspace*{\fill} \\}
    
    For \texttt{ground\ 3} with interdistance = 0.7

    \begin{Verbatim}[commandchars=\\\{\}]
{\color{incolor}In [{\color{incolor}9}]:} \PY{n}{available\PYZus{}spots} \PY{o}{=} \PY{n}{ground\PYZus{}3}\PY{o}{.}\PY{n}{arrayRGB\PYZus{}clusters}\PY{p}{(}\PY{p}{)}
        \PY{n}{\PYZus{}} \PY{o}{=} \PY{n}{plt}\PY{o}{.}\PY{n}{figure}\PY{p}{(} \PY{n}{figsize} \PY{o}{=} \PY{p}{(}\PY{l+m+mi}{7}\PY{p}{,} \PY{l+m+mi}{7}\PY{p}{)} \PY{p}{)}
        \PY{n}{\PYZus{}} \PY{o}{=} \PY{n}{plt}\PY{o}{.}\PY{n}{imshow}\PY{p}{(}\PY{n}{np}\PY{o}{.}\PY{n}{swapaxes}\PY{p}{(}\PY{n}{available\PYZus{}spots}\PY{p}{,}\PY{l+m+mi}{0}\PY{p}{,}\PY{l+m+mi}{1}\PY{p}{)}\PY{p}{,} \PY{n}{origin} \PY{o}{=} \PY{l+s+s1}{\PYZsq{}}\PY{l+s+s1}{lower}\PY{l+s+s1}{\PYZsq{}}\PY{p}{)}
        \PY{n}{\PYZus{}} \PY{o}{=} \PY{n}{plt}\PY{o}{.}\PY{n}{plot}\PY{p}{(}\PY{n}{ground\PYZus{}3}\PY{o}{.}\PY{n}{fp\PYZus{}center\PYZus{}x}\PY{p}{,} \PY{n}{ground\PYZus{}3}\PY{o}{.}\PY{n}{fp\PYZus{}center\PYZus{}y}\PY{p}{,} \PY{l+s+s1}{\PYZsq{}}\PY{l+s+s1}{+}\PY{l+s+s1}{\PYZsq{}}\PY{p}{,} \PY{n}{mew}\PY{o}{=}\PY{l+m+mi}{5}\PY{p}{,} \PY{n}{ms}\PY{o}{=}\PY{l+m+mi}{20}\PY{p}{)}
        \PY{n}{\PYZus{}} \PY{o}{=} \PY{n}{plt}\PY{o}{.}\PY{n}{plot}\PY{p}{(}\PY{n}{ground\PYZus{}3}\PY{o}{.}\PY{n}{sp\PYZus{}center\PYZus{}x}\PY{p}{,} \PY{n}{ground\PYZus{}3}\PY{o}{.}\PY{n}{sp\PYZus{}center\PYZus{}y}\PY{p}{,} \PY{l+s+s1}{\PYZsq{}}\PY{l+s+s1}{+}\PY{l+s+s1}{\PYZsq{}}\PY{p}{,} \PY{n}{mew}\PY{o}{=}\PY{l+m+mi}{5}\PY{p}{,} \PY{n}{ms}\PY{o}{=}\PY{l+m+mi}{20}\PY{p}{)}
\end{Verbatim}


    \begin{center}
    \adjustimage{max size={0.9\linewidth}{0.9\paperheight}}{output_20_0.png}
    \end{center}
    { \hspace*{\fill} \\}
    
    \hypertarget{first-population-of-food-and-sleep-patches}{%
\subsubsection{First population of food and sleep
patches}\label{first-population-of-food-and-sleep-patches}}

    For \texttt{ground\ 1} with interdistance = 0.

Food patches in green and sleep patches in lila. Individual in black.

    \begin{Verbatim}[commandchars=\\\{\}]
{\color{incolor}In [{\color{incolor}10}]:} \PY{n}{RGBdisplay} \PY{o}{=} \PY{n}{ground\PYZus{}1}\PY{o}{.}\PY{n}{arrayRGB\PYZus{}display}\PY{p}{(}\PY{p}{)}
         \PY{n}{\PYZus{}} \PY{o}{=} \PY{n}{plt}\PY{o}{.}\PY{n}{figure}\PY{p}{(} \PY{n}{figsize} \PY{o}{=} \PY{p}{(}\PY{l+m+mi}{9}\PY{p}{,} \PY{l+m+mi}{9}\PY{p}{)} \PY{p}{)}
         \PY{n}{\PYZus{}} \PY{o}{=} \PY{n}{plt}\PY{o}{.}\PY{n}{imshow}\PY{p}{(}\PY{n}{np}\PY{o}{.}\PY{n}{swapaxes}\PY{p}{(}\PY{n}{RGBdisplay}\PY{p}{,}\PY{l+m+mi}{0}\PY{p}{,}\PY{l+m+mi}{1}\PY{p}{)}\PY{p}{,} \PY{n}{origin} \PY{o}{=} \PY{l+s+s1}{\PYZsq{}}\PY{l+s+s1}{lower}\PY{l+s+s1}{\PYZsq{}}\PY{p}{)}
\end{Verbatim}


    \begin{center}
    \adjustimage{max size={0.9\linewidth}{0.9\paperheight}}{output_23_0.png}
    \end{center}
    { \hspace*{\fill} \\}
    
    For \texttt{ground\ 2} with interdistance = 0.35

    \begin{Verbatim}[commandchars=\\\{\}]
{\color{incolor}In [{\color{incolor}11}]:} \PY{n}{RGBdisplay} \PY{o}{=} \PY{n}{ground\PYZus{}2}\PY{o}{.}\PY{n}{arrayRGB\PYZus{}display}\PY{p}{(}\PY{p}{)}
         \PY{n}{\PYZus{}} \PY{o}{=} \PY{n}{plt}\PY{o}{.}\PY{n}{figure}\PY{p}{(} \PY{n}{figsize} \PY{o}{=} \PY{p}{(}\PY{l+m+mi}{9}\PY{p}{,} \PY{l+m+mi}{9}\PY{p}{)} \PY{p}{)}
         \PY{n}{\PYZus{}} \PY{o}{=} \PY{n}{plt}\PY{o}{.}\PY{n}{imshow}\PY{p}{(}\PY{n}{np}\PY{o}{.}\PY{n}{swapaxes}\PY{p}{(}\PY{n}{RGBdisplay}\PY{p}{,}\PY{l+m+mi}{0}\PY{p}{,}\PY{l+m+mi}{1}\PY{p}{)}\PY{p}{,} \PY{n}{origin} \PY{o}{=} \PY{l+s+s1}{\PYZsq{}}\PY{l+s+s1}{lower}\PY{l+s+s1}{\PYZsq{}}\PY{p}{)}
\end{Verbatim}


    \begin{center}
    \adjustimage{max size={0.9\linewidth}{0.9\paperheight}}{output_25_0.png}
    \end{center}
    { \hspace*{\fill} \\}
    
    For \texttt{ground\ 3} with interdistance = 0.7

    \begin{Verbatim}[commandchars=\\\{\}]
{\color{incolor}In [{\color{incolor}12}]:} \PY{n}{RGBdisplay} \PY{o}{=} \PY{n}{ground\PYZus{}3}\PY{o}{.}\PY{n}{arrayRGB\PYZus{}display}\PY{p}{(}\PY{p}{)}
         \PY{n}{\PYZus{}} \PY{o}{=} \PY{n}{plt}\PY{o}{.}\PY{n}{figure}\PY{p}{(} \PY{n}{figsize} \PY{o}{=} \PY{p}{(}\PY{l+m+mi}{9}\PY{p}{,} \PY{l+m+mi}{9}\PY{p}{)} \PY{p}{)}
         \PY{n}{\PYZus{}} \PY{o}{=} \PY{n}{plt}\PY{o}{.}\PY{n}{imshow}\PY{p}{(}\PY{n}{np}\PY{o}{.}\PY{n}{swapaxes}\PY{p}{(}\PY{n}{RGBdisplay}\PY{p}{,}\PY{l+m+mi}{0}\PY{p}{,}\PY{l+m+mi}{1}\PY{p}{)}\PY{p}{,} \PY{n}{origin} \PY{o}{=} \PY{l+s+s1}{\PYZsq{}}\PY{l+s+s1}{lower}\PY{l+s+s1}{\PYZsq{}}\PY{p}{)}
\end{Verbatim}


    \begin{center}
    \adjustimage{max size={0.9\linewidth}{0.9\paperheight}}{output_27_0.png}
    \end{center}
    { \hspace*{\fill} \\}
    
    \begin{Verbatim}[commandchars=\\\{\}]
{\color{incolor}In [{\color{incolor}13}]:} \PY{n}{display2D} \PY{o}{=} \PY{n}{ground\PYZus{}3}\PY{o}{.}\PY{n}{array2D\PYZus{}display}\PY{p}{(}\PY{p}{)}
         \PY{n}{\PYZus{}} \PY{o}{=} \PY{n}{plt}\PY{o}{.}\PY{n}{figure}\PY{p}{(} \PY{n}{figsize} \PY{o}{=} \PY{p}{(}\PY{l+m+mi}{9}\PY{p}{,} \PY{l+m+mi}{9}\PY{p}{)} \PY{p}{)}
         \PY{n}{\PYZus{}} \PY{o}{=} \PY{n}{plt}\PY{o}{.}\PY{n}{imshow}\PY{p}{(}\PY{n}{display2D}\PY{o}{.}\PY{n}{T}\PY{p}{,} \PY{l+s+s1}{\PYZsq{}}\PY{l+s+s1}{afmhot\PYZus{}r}\PY{l+s+s1}{\PYZsq{}}\PY{p}{,} \PY{n}{origin} \PY{o}{=} \PY{l+s+s1}{\PYZsq{}}\PY{l+s+s1}{lower}\PY{l+s+s1}{\PYZsq{}}\PY{p}{)}
\end{Verbatim}


    \begin{center}
    \adjustimage{max size={0.9\linewidth}{0.9\paperheight}}{output_28_0.png}
    \end{center}
    { \hspace*{\fill} \\}
    
    \hypertarget{testing-movement-of-agent}{%
\subsection{Testing movement of agent}\label{testing-movement-of-agent}}

In this section, for the \texttt{ground\ 3} (interdistance factor = 0.7)
the movement of the agent is tested.

Five steps are executed and for each step it is shown:

\begin{itemize}
\tightlist
\item
  Timestep
\item
  Position(x,y)
\item
  Direction of movement (integer encoding, and translation)
\end{itemize}

    \begin{Verbatim}[commandchars=\\\{\}]
{\color{incolor}In [{\color{incolor}14}]:} \PY{n}{ground\PYZus{}3}\PY{o}{.}\PY{n}{schedule}\PY{o}{.}\PY{n}{time} \PY{p}{,} \PY{n}{ground\PYZus{}3}\PY{o}{.}\PY{n}{schedule}\PY{o}{.}\PY{n}{agents}\PY{p}{[}\PY{l+m+mi}{0}\PY{p}{]}\PY{o}{.}\PY{n}{pos} \PY{p}{,} 
         \PY{n}{ground\PYZus{}3}\PY{o}{.}\PY{n}{schedule}\PY{o}{.}\PY{n}{agents}\PY{p}{[}\PY{l+m+mi}{0}\PY{p}{]}\PY{o}{.}\PY{n}{direction} \PY{p}{,} \PY{n}{ground\PYZus{}3}\PY{o}{.}\PY{n}{schedule}\PY{o}{.}\PY{n}{agents}\PY{p}{[}\PY{l+m+mi}{0}\PY{p}{]}\PY{o}{.}\PY{n}{lookingto}
\end{Verbatim}


\begin{Verbatim}[commandchars=\\\{\}]
{\color{outcolor}Out[{\color{outcolor}14}]:} (0, (22, 31))
\end{Verbatim}
            
\begin{Verbatim}[commandchars=\\\{\}]
{\color{outcolor}Out[{\color{outcolor}14}]:} (0, 'RIGHT')
\end{Verbatim}
            
    \begin{Verbatim}[commandchars=\\\{\}]
{\color{incolor}In [{\color{incolor}15}]:} \PY{n}{ground\PYZus{}3}\PY{o}{.}\PY{n}{step}\PY{p}{(}\PY{p}{)}
         \PY{n}{ground\PYZus{}3}\PY{o}{.}\PY{n}{schedule}\PY{o}{.}\PY{n}{time} \PY{p}{,} \PY{n}{ground\PYZus{}3}\PY{o}{.}\PY{n}{schedule}\PY{o}{.}\PY{n}{agents}\PY{p}{[}\PY{l+m+mi}{0}\PY{p}{]}\PY{o}{.}\PY{n}{pos} \PY{p}{,} 
         \PY{n}{ground\PYZus{}3}\PY{o}{.}\PY{n}{schedule}\PY{o}{.}\PY{n}{agents}\PY{p}{[}\PY{l+m+mi}{0}\PY{p}{]}\PY{o}{.}\PY{n}{direction} \PY{p}{,} \PY{n}{ground\PYZus{}3}\PY{o}{.}\PY{n}{schedule}\PY{o}{.}\PY{n}{agents}\PY{p}{[}\PY{l+m+mi}{0}\PY{p}{]}\PY{o}{.}\PY{n}{lookingto}
\end{Verbatim}


\begin{Verbatim}[commandchars=\\\{\}]
{\color{outcolor}Out[{\color{outcolor}15}]:} (1, (23, 31))
\end{Verbatim}
            
\begin{Verbatim}[commandchars=\\\{\}]
{\color{outcolor}Out[{\color{outcolor}15}]:} (1, 'RIGHT')
\end{Verbatim}
            
    \begin{Verbatim}[commandchars=\\\{\}]
{\color{incolor}In [{\color{incolor}16}]:} \PY{n}{ground\PYZus{}3}\PY{o}{.}\PY{n}{step}\PY{p}{(}\PY{p}{)}
         \PY{n}{ground\PYZus{}3}\PY{o}{.}\PY{n}{schedule}\PY{o}{.}\PY{n}{time} \PY{p}{,} \PY{n}{ground\PYZus{}3}\PY{o}{.}\PY{n}{schedule}\PY{o}{.}\PY{n}{agents}\PY{p}{[}\PY{l+m+mi}{0}\PY{p}{]}\PY{o}{.}\PY{n}{pos} \PY{p}{,} 
         \PY{n}{ground\PYZus{}3}\PY{o}{.}\PY{n}{schedule}\PY{o}{.}\PY{n}{agents}\PY{p}{[}\PY{l+m+mi}{0}\PY{p}{]}\PY{o}{.}\PY{n}{direction} \PY{p}{,} \PY{n}{ground\PYZus{}3}\PY{o}{.}\PY{n}{schedule}\PY{o}{.}\PY{n}{agents}\PY{p}{[}\PY{l+m+mi}{0}\PY{p}{]}\PY{o}{.}\PY{n}{lookingto}
\end{Verbatim}


\begin{Verbatim}[commandchars=\\\{\}]
{\color{outcolor}Out[{\color{outcolor}16}]:} (2, (24, 31))
\end{Verbatim}
            
\begin{Verbatim}[commandchars=\\\{\}]
{\color{outcolor}Out[{\color{outcolor}16}]:} (0, 'RIGHT')
\end{Verbatim}
            
    \begin{Verbatim}[commandchars=\\\{\}]
{\color{incolor}In [{\color{incolor}17}]:} \PY{n}{ground\PYZus{}3}\PY{o}{.}\PY{n}{step}\PY{p}{(}\PY{p}{)}
         \PY{n}{ground\PYZus{}3}\PY{o}{.}\PY{n}{schedule}\PY{o}{.}\PY{n}{time} \PY{p}{,} \PY{n}{ground\PYZus{}3}\PY{o}{.}\PY{n}{schedule}\PY{o}{.}\PY{n}{agents}\PY{p}{[}\PY{l+m+mi}{0}\PY{p}{]}\PY{o}{.}\PY{n}{pos} \PY{p}{,} 
         \PY{n}{ground\PYZus{}3}\PY{o}{.}\PY{n}{schedule}\PY{o}{.}\PY{n}{agents}\PY{p}{[}\PY{l+m+mi}{0}\PY{p}{]}\PY{o}{.}\PY{n}{direction} \PY{p}{,} \PY{n}{ground\PYZus{}3}\PY{o}{.}\PY{n}{schedule}\PY{o}{.}\PY{n}{agents}\PY{p}{[}\PY{l+m+mi}{0}\PY{p}{]}\PY{o}{.}\PY{n}{lookingto}
\end{Verbatim}


\begin{Verbatim}[commandchars=\\\{\}]
{\color{outcolor}Out[{\color{outcolor}17}]:} (3, (25, 31))
\end{Verbatim}
            
\begin{Verbatim}[commandchars=\\\{\}]
{\color{outcolor}Out[{\color{outcolor}17}]:} (0, 'RIGHT')
\end{Verbatim}
            
    \begin{Verbatim}[commandchars=\\\{\}]
{\color{incolor}In [{\color{incolor}18}]:} \PY{n}{ground\PYZus{}3}\PY{o}{.}\PY{n}{step}\PY{p}{(}\PY{p}{)}
         \PY{n}{ground\PYZus{}3}\PY{o}{.}\PY{n}{schedule}\PY{o}{.}\PY{n}{time} \PY{p}{,} \PY{n}{ground\PYZus{}3}\PY{o}{.}\PY{n}{schedule}\PY{o}{.}\PY{n}{agents}\PY{p}{[}\PY{l+m+mi}{0}\PY{p}{]}\PY{o}{.}\PY{n}{pos} \PY{p}{,} 
         \PY{n}{ground\PYZus{}3}\PY{o}{.}\PY{n}{schedule}\PY{o}{.}\PY{n}{agents}\PY{p}{[}\PY{l+m+mi}{0}\PY{p}{]}\PY{o}{.}\PY{n}{direction} \PY{p}{,} \PY{n}{ground\PYZus{}3}\PY{o}{.}\PY{n}{schedule}\PY{o}{.}\PY{n}{agents}\PY{p}{[}\PY{l+m+mi}{0}\PY{p}{]}\PY{o}{.}\PY{n}{lookingto}
\end{Verbatim}


\begin{Verbatim}[commandchars=\\\{\}]
{\color{outcolor}Out[{\color{outcolor}18}]:} (4, (26, 31))
\end{Verbatim}
            
\begin{Verbatim}[commandchars=\\\{\}]
{\color{outcolor}Out[{\color{outcolor}18}]:} (0, 'RIGHT')
\end{Verbatim}
            
    \begin{Verbatim}[commandchars=\\\{\}]
{\color{incolor}In [{\color{incolor}19}]:} \PY{n}{ground\PYZus{}3}\PY{o}{.}\PY{n}{step}\PY{p}{(}\PY{p}{)}
         \PY{n}{ground\PYZus{}3}\PY{o}{.}\PY{n}{schedule}\PY{o}{.}\PY{n}{time} \PY{p}{,} \PY{n}{ground\PYZus{}3}\PY{o}{.}\PY{n}{schedule}\PY{o}{.}\PY{n}{agents}\PY{p}{[}\PY{l+m+mi}{0}\PY{p}{]}\PY{o}{.}\PY{n}{pos} \PY{p}{,} \PY{n}{ground\PYZus{}3}\PY{o}{.}\PY{n}{schedule}\PY{o}{.}\PY{n}{agents}\PY{p}{[}\PY{l+m+mi}{0}\PY{p}{]}\PY{o}{.}\PY{n}{direction} \PY{p}{,} \PY{n}{ground\PYZus{}3}\PY{o}{.}\PY{n}{schedule}\PY{o}{.}\PY{n}{agents}\PY{p}{[}\PY{l+m+mi}{0}\PY{p}{]}\PY{o}{.}\PY{n}{lookingto}
\end{Verbatim}


\begin{Verbatim}[commandchars=\\\{\}]
{\color{outcolor}Out[{\color{outcolor}19}]:} (5, (27, 31), 1, 'RIGHT')
\end{Verbatim}
            
    \begin{Verbatim}[commandchars=\\\{\}]
{\color{incolor}In [{\color{incolor}20}]:} \PY{n}{RGBdisplay} \PY{o}{=} \PY{n}{ground\PYZus{}3}\PY{o}{.}\PY{n}{arrayRGB\PYZus{}display}\PY{p}{(}\PY{p}{)}
         \PY{n}{\PYZus{}} \PY{o}{=} \PY{n}{plt}\PY{o}{.}\PY{n}{figure}\PY{p}{(} \PY{n}{figsize} \PY{o}{=} \PY{p}{(}\PY{l+m+mi}{9}\PY{p}{,} \PY{l+m+mi}{9}\PY{p}{)} \PY{p}{)}
         \PY{n}{\PYZus{}} \PY{o}{=} \PY{n}{plt}\PY{o}{.}\PY{n}{imshow}\PY{p}{(}\PY{n}{np}\PY{o}{.}\PY{n}{swapaxes}\PY{p}{(}\PY{n}{RGBdisplay}\PY{p}{,}\PY{l+m+mi}{0}\PY{p}{,}\PY{l+m+mi}{1}\PY{p}{)}\PY{p}{,} \PY{n}{origin} \PY{o}{=} \PY{l+s+s1}{\PYZsq{}}\PY{l+s+s1}{lower}\PY{l+s+s1}{\PYZsq{}}\PY{p}{)}
\end{Verbatim}


    \begin{center}
    \adjustimage{max size={0.9\linewidth}{0.9\paperheight}}{output_36_0.png}
    \end{center}
    { \hspace*{\fill} \\}
    
    \hypertarget{testing-data-collection}{%
\subsection{Testing data collection}\label{testing-data-collection}}

Now, the simulation is run for 495 additional steps. Then, the grid is
shown and the scatter plot of the Fitness.

    \begin{Verbatim}[commandchars=\\\{\}]
{\color{incolor}In [{\color{incolor}21}]:} \PY{k}{for} \PY{n}{i} \PY{o+ow}{in} \PY{n+nb}{range}\PY{p}{(}\PY{l+m+mi}{495}\PY{p}{)}\PY{p}{:}
             \PY{n}{ground\PYZus{}3}\PY{o}{.}\PY{n}{step}\PY{p}{(}\PY{p}{)}
\end{Verbatim}


    \begin{Verbatim}[commandchars=\\\{\}]
{\color{incolor}In [{\color{incolor}22}]:} \PY{n}{ground\PYZus{}3}\PY{o}{.}\PY{n}{schedule}\PY{o}{.}\PY{n}{time} \PY{p}{,} \PY{n}{ground\PYZus{}3}\PY{o}{.}\PY{n}{schedule}\PY{o}{.}\PY{n}{agents}\PY{p}{[}\PY{l+m+mi}{0}\PY{p}{]}\PY{o}{.}\PY{n}{pos} \PY{p}{,} 
         \PY{n}{ground\PYZus{}3}\PY{o}{.}\PY{n}{schedule}\PY{o}{.}\PY{n}{agents}\PY{p}{[}\PY{l+m+mi}{0}\PY{p}{]}\PY{o}{.}\PY{n}{direction} \PY{p}{,} \PY{n}{ground\PYZus{}3}\PY{o}{.}\PY{n}{schedule}\PY{o}{.}\PY{n}{agents}\PY{p}{[}\PY{l+m+mi}{0}\PY{p}{]}\PY{o}{.}\PY{n}{lookingto}
\end{Verbatim}


\begin{Verbatim}[commandchars=\\\{\}]
{\color{outcolor}Out[{\color{outcolor}22}]:} (500, (0, 0))
\end{Verbatim}
            
\begin{Verbatim}[commandchars=\\\{\}]
{\color{outcolor}Out[{\color{outcolor}22}]:} (13, 'LEFT')
\end{Verbatim}
            
    \begin{Verbatim}[commandchars=\\\{\}]
{\color{incolor}In [{\color{incolor}23}]:} \PY{n}{RGBdisplay} \PY{o}{=} \PY{n}{ground\PYZus{}3}\PY{o}{.}\PY{n}{arrayRGB\PYZus{}display}\PY{p}{(}\PY{p}{)}
         \PY{n}{\PYZus{}} \PY{o}{=} \PY{n}{plt}\PY{o}{.}\PY{n}{figure}\PY{p}{(} \PY{n}{figsize} \PY{o}{=} \PY{p}{(}\PY{l+m+mi}{9}\PY{p}{,} \PY{l+m+mi}{9}\PY{p}{)} \PY{p}{)}
         \PY{n}{\PYZus{}} \PY{o}{=} \PY{n}{plt}\PY{o}{.}\PY{n}{imshow}\PY{p}{(}\PY{n}{np}\PY{o}{.}\PY{n}{swapaxes}\PY{p}{(}\PY{n}{RGBdisplay}\PY{p}{,}\PY{l+m+mi}{0}\PY{p}{,}\PY{l+m+mi}{1}\PY{p}{)}\PY{p}{,} \PY{n}{origin} \PY{o}{=} \PY{l+s+s1}{\PYZsq{}}\PY{l+s+s1}{lower}\PY{l+s+s1}{\PYZsq{}}\PY{p}{)}
\end{Verbatim}


    \begin{center}
    \adjustimage{max size={0.9\linewidth}{0.9\paperheight}}{output_40_0.png}
    \end{center}
    { \hspace*{\fill} \\}
    
    \begin{Verbatim}[commandchars=\\\{\}]
{\color{incolor}In [{\color{incolor}24}]:} \PY{n}{agents} \PY{o}{=} \PY{n}{ground\PYZus{}3}\PY{o}{.}\PY{n}{datacollector}\PY{o}{.}\PY{n}{get\PYZus{}agent\PYZus{}vars\PYZus{}dataframe}\PY{p}{(}\PY{p}{)}
         \PY{n}{Agent1\PYZus{}fitness} \PY{o}{=} \PY{n}{agents}\PY{o}{.}\PY{n}{xs}\PY{p}{(}\PY{l+m+mi}{1}\PY{p}{,} \PY{n}{level}\PY{o}{=}\PY{l+s+s2}{\PYZdq{}}\PY{l+s+s2}{AgentID}\PY{l+s+s2}{\PYZdq{}}\PY{p}{)}
         \PY{n}{\PYZus{}} \PY{o}{=} \PY{n}{Agent1\PYZus{}fitness}\PY{o}{.}\PY{n}{Fitness}\PY{o}{.}\PY{n}{plot}\PY{p}{(}\PY{p}{)}
\end{Verbatim}


    \begin{center}
    \adjustimage{max size={0.9\linewidth}{0.9\paperheight}}{output_41_0.png}
    \end{center}
    { \hspace*{\fill} \\}
    
    Now, the simulation will be run for 3500, 3500, and 3000 steps, and the
grid shown after each interval.

    \begin{Verbatim}[commandchars=\\\{\}]
{\color{incolor}In [{\color{incolor}25}]:} \PY{k}{for} \PY{n}{i} \PY{o+ow}{in} \PY{n+nb}{range}\PY{p}{(}\PY{l+m+mi}{3500}\PY{p}{)}\PY{p}{:}
             \PY{n}{ground\PYZus{}3}\PY{o}{.}\PY{n}{step}\PY{p}{(}\PY{p}{)}
\end{Verbatim}


    \begin{Verbatim}[commandchars=\\\{\}]
{\color{incolor}In [{\color{incolor}26}]:} \PY{n}{RGBdisplay} \PY{o}{=} \PY{n}{ground\PYZus{}3}\PY{o}{.}\PY{n}{arrayRGB\PYZus{}display}\PY{p}{(}\PY{p}{)}
         \PY{n}{\PYZus{}} \PY{o}{=} \PY{n}{plt}\PY{o}{.}\PY{n}{figure}\PY{p}{(} \PY{n}{figsize} \PY{o}{=} \PY{p}{(}\PY{l+m+mi}{9}\PY{p}{,} \PY{l+m+mi}{9}\PY{p}{)} \PY{p}{)}
         \PY{n}{\PYZus{}} \PY{o}{=} \PY{n}{plt}\PY{o}{.}\PY{n}{imshow}\PY{p}{(}\PY{n}{np}\PY{o}{.}\PY{n}{swapaxes}\PY{p}{(}\PY{n}{RGBdisplay}\PY{p}{,}\PY{l+m+mi}{0}\PY{p}{,}\PY{l+m+mi}{1}\PY{p}{)}\PY{p}{,} \PY{n}{origin} \PY{o}{=} \PY{l+s+s1}{\PYZsq{}}\PY{l+s+s1}{lower}\PY{l+s+s1}{\PYZsq{}}\PY{p}{)}
\end{Verbatim}


    \begin{center}
    \adjustimage{max size={0.9\linewidth}{0.9\paperheight}}{output_44_0.png}
    \end{center}
    { \hspace*{\fill} \\}
    
    \begin{Verbatim}[commandchars=\\\{\}]
{\color{incolor}In [{\color{incolor}27}]:} \PY{k}{for} \PY{n}{i} \PY{o+ow}{in} \PY{n+nb}{range}\PY{p}{(}\PY{l+m+mi}{3000}\PY{p}{)}\PY{p}{:}
             \PY{n}{ground\PYZus{}3}\PY{o}{.}\PY{n}{step}\PY{p}{(}\PY{p}{)}
         \PY{n}{RGBdisplay} \PY{o}{=} \PY{n}{ground\PYZus{}3}\PY{o}{.}\PY{n}{arrayRGB\PYZus{}display}\PY{p}{(}\PY{p}{)}
         \PY{n}{\PYZus{}} \PY{o}{=} \PY{n}{plt}\PY{o}{.}\PY{n}{figure}\PY{p}{(} \PY{n}{figsize} \PY{o}{=} \PY{p}{(}\PY{l+m+mi}{9}\PY{p}{,} \PY{l+m+mi}{9}\PY{p}{)} \PY{p}{)}
         \PY{n}{\PYZus{}} \PY{o}{=} \PY{n}{plt}\PY{o}{.}\PY{n}{imshow}\PY{p}{(}\PY{n}{np}\PY{o}{.}\PY{n}{swapaxes}\PY{p}{(}\PY{n}{RGBdisplay}\PY{p}{,}\PY{l+m+mi}{0}\PY{p}{,}\PY{l+m+mi}{1}\PY{p}{)}\PY{p}{,} \PY{n}{origin} \PY{o}{=} \PY{l+s+s1}{\PYZsq{}}\PY{l+s+s1}{lower}\PY{l+s+s1}{\PYZsq{}}\PY{p}{)}
\end{Verbatim}


    \begin{center}
    \adjustimage{max size={0.9\linewidth}{0.9\paperheight}}{output_45_0.png}
    \end{center}
    { \hspace*{\fill} \\}
    
    \begin{Verbatim}[commandchars=\\\{\}]
{\color{incolor}In [{\color{incolor}28}]:} \PY{k}{for} \PY{n}{i} \PY{o+ow}{in} \PY{n+nb}{range}\PY{p}{(}\PY{l+m+mi}{3000}\PY{p}{)}\PY{p}{:}
             \PY{n}{ground\PYZus{}3}\PY{o}{.}\PY{n}{step}\PY{p}{(}\PY{p}{)}
         \PY{n}{RGBdisplay} \PY{o}{=} \PY{n}{ground\PYZus{}3}\PY{o}{.}\PY{n}{arrayRGB\PYZus{}display}\PY{p}{(}\PY{p}{)}
         \PY{n}{\PYZus{}} \PY{o}{=} \PY{n}{plt}\PY{o}{.}\PY{n}{figure}\PY{p}{(} \PY{n}{figsize} \PY{o}{=} \PY{p}{(}\PY{l+m+mi}{9}\PY{p}{,} \PY{l+m+mi}{9}\PY{p}{)} \PY{p}{)}
         \PY{n}{\PYZus{}} \PY{o}{=} \PY{n}{plt}\PY{o}{.}\PY{n}{imshow}\PY{p}{(}\PY{n}{np}\PY{o}{.}\PY{n}{swapaxes}\PY{p}{(}\PY{n}{RGBdisplay}\PY{p}{,}\PY{l+m+mi}{0}\PY{p}{,}\PY{l+m+mi}{1}\PY{p}{)}\PY{p}{,} \PY{n}{origin} \PY{o}{=} \PY{l+s+s1}{\PYZsq{}}\PY{l+s+s1}{lower}\PY{l+s+s1}{\PYZsq{}}\PY{p}{)}
\end{Verbatim}


    \begin{center}
    \adjustimage{max size={0.9\linewidth}{0.9\paperheight}}{output_46_0.png}
    \end{center}
    { \hspace*{\fill} \\}
    
    From the automated data collectors, information has been captured about
the agent for each timestep. As an example, it is shown:

\begin{itemize}
\tightlist
\item
  Mode of the individual (10 represents \texttt{feeding\ mode}, -10 is
  \texttt{sleeping\ mode}, and 0 is \texttt{roaming\ mode})
\item
  Fitness from food energy.
\item
  Fitnes from sleep energy.
\item
  Average fitness of the individual.
\end{itemize}

    \begin{Verbatim}[commandchars=\\\{\}]
{\color{incolor}In [{\color{incolor}29}]:} \PY{n}{agents} \PY{o}{=} \PY{n}{ground\PYZus{}3}\PY{o}{.}\PY{n}{datacollector}\PY{o}{.}\PY{n}{get\PYZus{}agent\PYZus{}vars\PYZus{}dataframe}\PY{p}{(}\PY{p}{)}
         \PY{n}{agent1\PYZus{}data} \PY{o}{=} \PY{n}{agents}\PY{o}{.}\PY{n}{xs}\PY{p}{(}\PY{l+m+mi}{1}\PY{p}{,} \PY{n}{level}\PY{o}{=}\PY{l+s+s2}{\PYZdq{}}\PY{l+s+s2}{AgentID}\PY{l+s+s2}{\PYZdq{}}\PY{p}{)}
         \PY{n}{\PYZus{}} \PY{o}{=} \PY{n}{agent1\PYZus{}data}\PY{o}{.}\PY{n}{Mode}\PY{o}{.}\PY{n}{plot}\PY{p}{(}\PY{p}{)}
\end{Verbatim}


    \begin{center}
    \adjustimage{max size={0.9\linewidth}{0.9\paperheight}}{output_48_0.png}
    \end{center}
    { \hspace*{\fill} \\}
    
    \begin{Verbatim}[commandchars=\\\{\}]
{\color{incolor}In [{\color{incolor}30}]:} \PY{n}{\PYZus{}} \PY{o}{=} \PY{n}{agent1\PYZus{}data}\PY{o}{.}\PY{n}{Food\PYZus{}fitness}\PY{o}{.}\PY{n}{plot}\PY{p}{(}\PY{p}{)}
\end{Verbatim}


    \begin{center}
    \adjustimage{max size={0.9\linewidth}{0.9\paperheight}}{output_49_0.png}
    \end{center}
    { \hspace*{\fill} \\}
    
    \begin{Verbatim}[commandchars=\\\{\}]
{\color{incolor}In [{\color{incolor}31}]:} \PY{n}{\PYZus{}} \PY{o}{=} \PY{n}{agent1\PYZus{}data}\PY{o}{.}\PY{n}{Sleep\PYZus{}fitness}\PY{o}{.}\PY{n}{plot}\PY{p}{(}\PY{p}{)}
\end{Verbatim}


    \begin{center}
    \adjustimage{max size={0.9\linewidth}{0.9\paperheight}}{output_50_0.png}
    \end{center}
    { \hspace*{\fill} \\}
    
    \begin{Verbatim}[commandchars=\\\{\}]
{\color{incolor}In [{\color{incolor}32}]:} \PY{n}{\PYZus{}} \PY{o}{=} \PY{n}{agent1\PYZus{}data}\PY{o}{.}\PY{n}{Fitness}\PY{o}{.}\PY{n}{plot}\PY{p}{(}\PY{p}{)}
\end{Verbatim}


    \begin{center}
    \adjustimage{max size={0.9\linewidth}{0.9\paperheight}}{output_51_0.png}
    \end{center}
    { \hspace*{\fill} \\}
    

    % Add a bibliography block to the postdoc
    
    
    
    \end{document}
